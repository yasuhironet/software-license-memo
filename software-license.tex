\documentclass{jsarticle}
\title{ソフトウェアライセンス}
\author{}
\date{2024/01/04}

\usepackage{url}

\begin{document}
\maketitle

\begin{itemize}

\item Copyright: 著作権
\item License: 利用許諾

ソフトウェアの著作権を持つものは、そのソフトウェアのユーザへの
利用許諾を自由に決めることができる。

\item Dual-Licensing

複数の利用許諾(license)で
ユーザに提供することを、dual-license と呼ぶ。
例えば、MySQL は dual-licensing していて、コミュニティ版と商用版がある。
X11も有名。GitLab CE/EE も。

\item Free Software / Open-source Software
\begin{itemize}
\item \url{https://www.gnu.org/philosophy/free-sw.html}
\end{itemize}

タダ(無償)という意味ではない、真に自由なフリーソフトウェアとは
何か、という思想。ユーザがプログラム(開発)の恩恵を得られる権利を
維持することを自由と呼ぶ、というような思想。
ソースが無いと過去の開発の恩恵を得られないので、ソースはオープンに
しよう、という意味で、フリーソフトウェアとオープンソースソフトウェアの
考え方はほぼ同義、ほぼ同一。

\item Copyleft
\begin{itemize}
\item \url{https://www.gnu.org/licenses/copyleft.ja.html}
\end{itemize}

プログラムの著者は著作権(copyrights)を主張せず、
パブリックドメインにする(公共に所属させる)べきである、とする思想。
Free Software / Open-source Software の考えの根底にある思想。
copyright -- all rights reserved.(著作権 -- 全ての権利は留保されている)を
揶揄した、
copyleft -- all rights reversed. (コピーレフト -- 全ての右は逆転されている)
に由来する。
ユーザの利用できる権利を守るため、該当プログラムを利用した派生物は全て、
ソースを公開するなどの同じ制限と利用許諾にしなければならない。
GPL(GNU Public License)が有名。
ユーザの権利を守ろうとする余り、プログラムの派生物にも同じ
コピーレフト性という強い制限を求めるため、GPL 汚染やライセンス感染と呼ばれる。
開発物のソースをすべて公開しなければならないという制限が、
多くのビジネスモデルと適合しない、と言われる。

\item Permissiveライセンス
\begin{itemize}
\item \url{https://opensource.org/faq/#permissive}
\end{itemize}

GPLのような制限の強いコピーレフトライセンスに対して、
MITやBSD 2/3-Clause の寛容なライセンスは Permissive ライセンスと呼ばれる。
このライセンスのソフトウェアも、正しく Free Software / Open-source Software
であるとされる。
著作権表示を消すな、大学名を販促に使うな、というような
緩い制限の2/3項目を守れば、商用や改変・再配布を含め、
どのような利用もできるライセンス。Apacheライセンスもこれ。

\item Selling Exceptions
\begin{itemize}
\item \url{https://www.gnu.org/philosophy/selling-exceptions.en.html}
\end{itemize}

真に自由なコピーレフトライセンスを使って、
ビジネス(商用利用)ができるようにしよう、
という、GNUの祖ストールマン(RMS)の例外規定。
GNUより前からあるX11を許すためには、、、みたいなこと言ってる。

\item Commons Clause
\begin{itemize}
\item \url{https://commonsclause.com/}
\end{itemize}

オープンソースライセンスの前段にこれを付ければ、
商用利用だけを禁止できるのでは?という提案。
かなり批判された模様。

\item Open Core
\begin{itemize}
\item \url{https://en.wikipedia.org/wiki/Open-core_model}
\end{itemize}

dual-licensing でビジネス・商用利用もしつつ、
ベース部分を Open-source software とするビジネスモデル。
Kafka, Cassandra, Oracle/MySQL, Eucalyptus, GitLab, Redis
などが該当するらしい。

\item Source-available Software
\begin{itemize}
\item \url{https://en.wikipedia.org/wiki/Source-available_software}
\end{itemize}

ソースを公開しているが、Free Software / Open-source Software に
そぐわないライセンスでソフトウェア開発・提供しているモデル。
Elasticがopen-core からこちらに移行した模様。
Redisのplug-in部分もこのモデルのライセンス(Redis Source Available License)。
HasiCorp/Terraformもこのモデル(Business Source License (BSL))。
MongoDBのSSPLもソース公開だがフリー・オープンソースでない商用ライセンス
であり、このモデル。

\item 伽藍とバザール
\begin{itemize}
\item \url{http://www.catb.org/~esr/writings/cathedral-bazaar/}
\end{itemize}

荘厳な伽藍を建築するのではなくて、バザーのように人々に勝手に
やらせよう、というソフトウェア開発の方針の思想。
FRRouting (Free-range Routing: 放し飼いのルーティング)は
バザーと同じような思想でやっていると思われる。

\item LGPL
\begin{itemize}
\item \url{https://www.gnu.org/licenses/lgpl-3.0.html}
\end{itemize}

Link するだけのプログラムにはGPLを求めないような、
Lesser(劣等)GPL。ライブラリのために生まれた。

\item Creative Commons
\begin{itemize}
\item \url{https://creativecommons.jp/licenses/}
\end{itemize}

文書のためのライセンス体系、部品になっていて組み合わせで使える。
CC BY-NC-ND(BY: 著作権表示、NC: 非商用、ND: 改変禁止)など。
FAQで、ソフトウェアにCC付けれるが、お勧めしない、と書いてある。

\item PolyForm
\begin{itemize}
\item \url{https://polyformproject.org/}
\end{itemize}

ソフトウェアに適応できるCCのようなものらしい。
非商用、とかある模様。

\end{itemize}

\end{document}

