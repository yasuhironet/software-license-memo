\documentclass{jsarticle}
\title{ソフトウェアライセンスのメモ}
\author{}
\date{2024/01/06}

\usepackage{url}

\newcommand{\copyrightname}{著作権}
\newcommand{\copyrightalt}{コピーライト}
\newcommand{\license}{利用許諾}
\newcommand{\licensealt}{ライセンス}
\newcommand{\duallicense}{デュアル・ライセンス}
\newcommand{\derivative}{派生物}
\newcommand{\derivativealt}{二次的著作物}
\newcommand{\freesoftware}{フリーソフトウェア}
\newcommand{\opensourcesoftware}{オープンソースソフトウェア}
\newcommand{\copyleft}{コピーレフト}
\newcommand{\freesoftoss}{\freesoftware{}/\opensourcesoftware{}}

\begin{document}
\maketitle

\begin{itemize}

\item \copyrightname{}(\copyrightalt{}(Copyright)と同義)
\item \license{}(\licensealt{}(License)と同義)

ソフトウェアの\copyrightname{}(\copyrightalt{})を持つものは、
そのソフトウェアのユーザへの\license{}(\licensealt{})を
自由に決めることができる。

\item \duallicense{}

複数の\license(\licensealt{})で
プログラムをユーザに提供することを、\duallicense{} と呼ぶ。
例えば、MySQL は \duallicense{} していて、コミュニティ版と商用版がある。
X11も有名。GitLab CE/EE も。

\item \derivative{}(\derivativealt{}(Derivative works)と同義)

プログラムを改変し再配布する場合、
元のプログラムを原著作物(Original work)、
改変後の著作物を\derivative{}と呼ぶ。

\item パブリックドメイン

ソフトウェアをパブリックドメイン(公共)にする、ということは、
著作者が著作権を放棄し、誰でもどのような利用も許諾される
状態にする、ということ。

\item \freesoftware{}(Free Software)
\item \opensourcesoftware{}(Open-source Software(OSS))
\begin{itemize}
\item \url{https://www.gnu.org/philosophy/free-sw.html}
\end{itemize}

タダ(無償)という意味ではない、真に自由な(フリーな)ソフトウェアとは
何か、という思想。ユーザがプログラム(開発)の恩恵を得られる権利を
維持することを自由と呼ぶ、というような思想。
ソースが無いと過去の開発の恩恵を得られないので、ソースはオープンに
しよう、という意味で、\freesoftware{}と\opensourcesoftware{}の
考え方はほぼ同義、ほぼ同一。

\item \copyleft{}
\begin{itemize}
\item \url{https://www.gnu.org/licenses/copyleft.ja.html}
\end{itemize}

ユーザがプログラムを自由に利用できる権利を守るために、
\derivative{}の\license{}にも強い制限を課し、
1) ソースを公開しなければならない、
2) 同じ\license{}を使わなければならない、とする思想。
\freesoftoss{} の考えの根底にある思想。
copyright -- all rights reserved.(著作権 -- 全ての権利は留保されている)を
揶揄した、
copyleft -- all rights reversed. (コピーレフト -- 全ての右は逆転されている)
に由来する。
ユーザの利用できる権利を守るため、該当プログラムを利用した派生物は全て、
ソースを公開するなどの同じ制限と利用許諾にしなければならない。
GPL(GNU Public License)が有名。
ユーザの権利を守ろうとする余り、プログラムの派生物にも同じ
コピーレフト性という強い制限を求めるため、GPL 汚染やライセンス感染と呼ばれる。
開発物のソースをすべて公開しなければならないという制限が、
多くのビジネスモデルと適合しない、と言われる。

\item Permissiveライセンス
\begin{itemize}
\item \url{https://opensource.org/faq/#permissive}
\end{itemize}

GPLのような制限の強いコピーレフトライセンスに対して、
MITやBSD 2/3-Clause の寛容なライセンスは Permissive ライセンスと呼ばれる。
このライセンスのソフトウェアも、
正しく\freesoftoss{}であるとされる。
著作権表示を消すな、大学名を販促に使うな、というような
緩い制限の2/3項目を守れば、商用や改変・再配布を含め、
どのような利用もできるライセンス。Apacheライセンスもこれ。

\item Selling Exceptions
\begin{itemize}
\item \url{https://www.gnu.org/philosophy/selling-exceptions.en.html}
\end{itemize}

真に自由なコピーレフトライセンスを使って、
ビジネス(商用利用)ができるようにしよう、
という、GNUの祖ストールマン(RMS)の例外規定。
GNUより前からあるX11を許すためには、、、という記述が見られる。

\item Commons Clause
\begin{itemize}
\item \url{https://commonsclause.com/}
\end{itemize}

オープンソースライセンスの前段にこれを付ければ、
商用利用だけを禁止できるのでは?という提案。
かなり批判された模様。

\item Open-coreモデル
\begin{itemize}
\item \url{https://en.wikipedia.org/wiki/Open-core_model}
\end{itemize}

\duallicense{}でビジネス・商用利用もしつつ、
ベース部分を\opensourcesoftware{}とするビジネスモデル。
Kafka, Cassandra, Oracle/MySQL, Eucalyptus, GitLab, Redis
などが該当するらしい。

\item Source-available Software
\begin{itemize}
\item \url{https://en.wikipedia.org/wiki/Source-available_software}
\end{itemize}

ソースを公開しているが、\freesoftoss{}に
そぐわないライセンスでソフトウェア開発・提供しているモデル。
Elasticが Open-core からこちらに移行した模様。
Redisのplug-in部分もこのモデルのライセンス(Redis Source Available License)。
HasiCorp/Terraformもこのモデル(Business Source License(BSL))。
MongoDBのSSPLもソース公開だがフリー・オープンソースでない商用ライセンス
であり、このモデル。

\item 伽藍とバザール
\begin{itemize}
\item \url{https://www.aozora.gr.jp/cards/000029/card227.html}
\end{itemize}

荘厳な伽藍を建築するのではなくて、バザーのように人々に勝手に
やらせよう、というソフトウェア開発の方針の思想。
FRRouting(Free-range(放し飼いの) Routing)は
バザーと同じような思想で開発されている。

\item LGPL
\begin{itemize}
\item \url{https://www.gnu.org/licenses/lgpl-3.0.html}
\end{itemize}

Link するだけのプログラムにはGPLを求めないような、
Lesser(劣等)GPL。ライブラリのために生まれた。

\item Creative Commons
\begin{itemize}
\item \url{https://creativecommons.jp/licenses/}
\end{itemize}

文書のためのライセンス体系、部品になっていて組み合わせで使える。
CC BY-NC-ND(BY: 著作権表示、NC: 非商用、ND: 改変禁止)など。
FAQで、ソフトウェアにCCを付与することは可能だが、お勧めしない、と書いてある。

\item PolyForm
\begin{itemize}
\item \url{https://polyformproject.org/}
\end{itemize}

ソフトウェアに適応できるCCのようなものらしい。
非商用、とかある模様。

\end{itemize}

\end{document}

